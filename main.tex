\documentclass[10pt]{article}
\usepackage{graphicx} % Required for inserting images
\usepackage{geometry}
\usepackage{amsmath}
\usepackage{amsfonts}

\geometry{
    a4paper,
    left=20mm,
    right=20mm,
    top=20mm,
    bottom=20mm
}

% commands
\newcommand{\deq}{\vcentcolon=}
\newcommand{\idd}{\text{đ}}
\newcommand{\nimplies}{\centernot\implies}
\newcommand{\vc}[1]{\boldsymbol{#1}}
\DeclareMathOperator*{\argmax}{arg\,max}
\DeclareMathOperator*{\argmin}{arg\,min}
\newcommand{\mat}[1]{\mathbf{#1}}
\newcommand{\hess}[1]{\vc{\nabla}^2 f(\vc{#1})}

\title{nla hw02}
\author{cs}

\begin{document}

\maketitle

\section{Question 1}

\subsection{(i)}
Let $0<\epsilon\leq1$, and
$$\mat{A}=\begin{bmatrix}
    1 & 0 \\
    0 & \epsilon
\end{bmatrix}.$$
We compute the condition numbers in $1$, $2$, and $\infty$-norms. So:
$$\mat{A}^{-1}=\begin{bmatrix}
    1 & 0 \\
    0 & 1/\epsilon
\end{bmatrix}.$$
The $1$-norm is max column sum:
$$||\mat{A}||_1=\max\{1,\epsilon\}=1$$
$$||\mat{A}^{-1}||_1=\max\{1,1/\epsilon\}=1/\epsilon$$
while the $2$-norm is the square root of the eigenvalue of $\mat{A}^T\mat{A}$
with the largest magnitude:
$$\mat{A}^T\mat{A}=\begin{bmatrix}
    1 & 0 \\
    0 & \epsilon^2
\end{bmatrix}$$
$$||\mat{A}||_2=1$$
$$(\mat{A}^{-1})^T\mat{A}^{-1}=\begin{bmatrix}
    1 & 0 \\
    0 & 1/\epsilon^2
\end{bmatrix}$$
$$||\mat{A}^{-1}||_2=1/\epsilon.$$
Finally the $\infty$-norm is the max row sum:
$$||\mat{A}||_\infty=1$$
$$||\mat{A}^{-1}||_\infty=1/\epsilon.$$
We are then able to compute the conditon numbers:
$$\kappa_1(\mat{A})=||\mat{A}||_1 ||\mat{A}^{-1}||_1=1/\epsilon$$
$$\kappa_2(\mat{A})=||\mat{A}||_2 ||\mat{A}^{-1}||_2=1/\epsilon$$
$$\kappa_\infty(\mat{A})
=||\mat{A}||_\infty ||\mat{A}^{-1}||_\infty=1/\epsilon.$$

\subsection{(ii)}
The condition numbers go to infinity as $\epsilon\rightarrow0$. This is in
compliance with the definition of condition numbers, since when
$\epsilon\rightarrow0$:
$$\mat{A}\rightarrow\begin{bmatrix}
    1 & 0 \\
    0 & 0
\end{bmatrix}$$
where matrix $\mat{A}$ becomes not invertible, and all $\kappa_p(\mat{A})\
=\infty$.

\section{Question 2}

\subsection{(i)}
Let $\mat{I}$ be the $n\times n$ identity matrix. Show that:
$$||\mat{I}||_p=1$$
for all $p$-norms.

\noindent\hrulefill

By definition, we have that:
\begin{align*}
    ||\mat{I}||_p
    &=\max_{\mat{x}\neq\mat{0}}
    \left\{\frac{||\mat{I}\mat{x}||_p}{||\mat{x}||_p}\right\} \\
    &=\max_{\mat{x}\neq\mat{0}}
    \left\{\frac{||\mat{x}||_p}{||\mat{x}||_p}\right\} \\
    &=1
\end{align*}
and this applies for any $p$.

\subsection{(ii)}
We show that if $\mathbf{A}$ is a $n\times n$ invertible matrix, then it's
conditional number must be greater or equal to one:
$$\kappa_p(\mathbf{A})\geq1$$
where this is true for any $p$-norm.

\noindent\hrulefill

By definition, we want to show that:
$$||\mathbf{A}||_p ||\mathbf{A}^{-1}||_p\geq1$$
where by the definition of the matrix $p$-norm $\mat{y}\neq\vc{0}$:
$$||\mat{A}||_p=
\max_{\mat{x}\neq\mat{0}}
\left\{\frac{||\mat{A}\mat{x}||_p}{||\mat{x}||_p}\right\}$$
$$||\mat{A}^{-1}||_p=
\max_{\mat{y}\neq\mat{0}}
\left\{\frac{||\mat{A}^{-1}\mat{y}||_p}{||\mat{y}||_p}\right\}$$
and so we can bound the left hand side of our goal by:
$$||\mathbf{A}||_p ||\mathbf{A}^{-1}||_p
\geq
\frac{||\mat{A}\mat{x}||_p}{||\mat{x}||_p}
\frac{||\mat{A}^{-1}\mat{y}||_p}{||\mat{y}||_p}$$
for any non-zero $\mat{x}$ and $\mat{y}$. But because matrix $\mat{A}$ is
invertible, it actually defines a bijective linear map between $\mat{x}$ and
$\mat{y}$ if we define relation $\mat{A}\mat{x}=\mat{y}$, which allows us to
conveniently write:
\begin{align*}
    ||\mathbf{A}||_p ||\mathbf{A}^{-1}||_p
    \geq
    \frac{||\mat{A}\mat{x}||_p}{||\mat{x}||_p}
    \frac{||\mat{A}^{-1}\mat{y}||_p}{||\mat{y}||_p}
    &=
    \frac{||\mat{y}||_p}{||\mat{A}^{-1}\mat{y}||_p}
    \frac{||\mat{A}^{-1}\mat{y}||_p}{||\mat{y}||_p} \\
    &=1.
\end{align*}

\end{document}